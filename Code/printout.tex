\documentclass{article}
\usepackage[utf8]{inputenc}
\usepackage[a4paper, margin=2cm]{geometry}
\usepackage{fancyhdr}
\usepackage{lastpage}
\usepackage{minted}

\usemintedstyle{vs}

\title{Aero Group A2-2 Odometry Code}

\makeatletter
\let\Title\@title
\makeatother

\pagestyle{fancy}
\lhead{}
\rhead{}
\rfoot{Page \textbf{\thepage} of \textbf{12}}
\renewcommand{\footrulewidth}{0.4pt}

\renewcommand{\theFancyVerbLine}{\sffamily \textcolor[rgb]{0.5,0.5,1.0}{\normalsize\arabic{FancyVerbLine}}}

\begin{document}

    \begin{center}
        \Large \textbf{\Title} \\[0.8cm]
        \hrule
        \normalsize
    \end{center}

    Code written by Duncan Hamill, Tom Griffiths, Ali Hajizadah, Robin Hannaford, and Felix Harris.

    \subsection*{Odometry.ino - Arduino logic}

    \inputminted[linenos]{cpp}{./Odometry/Odometry.ino}

    \subsection*{Course.cpp - Logic for completing the course}

    \inputminted[linenos]{cpp}{./Odometry/Course.cpp}

    \subsection*{Leg.cpp - defines code for completing a section of the course}

    \inputminted[linenos]{cpp}{./Odometry/Leg.cpp}

    \subsection*{Line.cpp - logic for driving a straight line}

    \inputminted[linenos]{cpp}{./Odometry/Line.cpp}

    \subsection*{Circle.cpp - logic for driving an arc of the course}

    \inputminted[linenos]{cpp}{./Odometry/Circle.cpp}

    \subsection*{encoderInteraction.cpp - functions for interacting with the MD25 encoders}

    \inputminted[linenos]{cpp}{./Odometry/encoderInteraction.cpp}

    \subsection*{defines.h - header including all definitions}

    \inputminted[linenos]{cpp}{./Odometry/defines.h}
    
\end{document}
